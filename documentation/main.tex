\documentclass[10.5pt]{article}
% compile with pdflatex deckball_bachelorarbeit.tex
\usepackage{amssymb}
\usepackage{color}
\usepackage{lipsum}
\usepackage{graphicx}
\usepackage[a4paper,bindingoffset=0.2in,%
    left=3.35cm,right=2.12cm,top=3.75cm,bottom=2.88cm,%
    footskip=.25in]{geometry}
\usepackage[T1]{fontenc}
\begin{document}

    \pagenumbering{gobble}
%{\color{red}Diese erste Seite bzw. Frontseite ist frei gestaltbar.}

%%%%%%%%%%% TITLE PAGE

    \begin{titlepage}
        \begin{figure}[t]
            \centering\includegraphics[width=0.5\textwidth]{HSLU2022logo}
            \label{fig:hslu-title}
        \end{figure}

        \begin{center}
            \textsc{\LARGE{Hochschule Luzern\\}}
            \textsc{ \LARGE{Computer Science Department\\ }}
            %\textnormal{ \LARGE{Corso di Laurea Triennale/Magistrale in ???\\}}
            \vspace{40mm}
            \textnormal
            Programming Concepts \& Paradigms\\
            \vspace{4mm}
            \fontsize{10mm}{7mm}\selectfont
            \textup{Rust}\\
        \end{center}

        \vspace{25mm}

        \begin{minipage}[t]{0.47\textwidth}
            \textnormal{\large{\bf Dozierende\\}}
            {\large Marcel Baumann \& Ruedi Arnold\\ \\}
        \end{minipage}\hfill\begin{minipage}[t]{0.47\textwidth}
                                \raggedleft
                                \textnormal{\large{\bf Studenten\\}}
                                {\large Roman Schilter \& Jan-Henrik Preuß}
        \end{minipage}

        \vspace{20mm}

        \centering{\large{Year 2024 }}

    \end{titlepage}

%    Anforderungen Bericht
%    § Qualität vor Quantität!
%    – Wenige Seiten (ca. 2-4) reichen durchaus, max. 5 (bei mehr
%    gibt's tendenziell Abzug), Inhaltsverzeichnis nicht nötig
%    § Fokus auf wichtige/ interessante/ spezielle Sprach-
%    Eigenschaften! (Was anders als bei Java?!...)
%    § Ergänzend zu den Folien
%    – Falls interessant/relevant kurze Infos zu Vision,
%    Geschichte & Verbreitung
%    – Hauptteil: Die Sprache vorstellen (Ihre 3 bis 7
%    Fokuspunkte, inkl. Verweise auf Ihren Demo-Code)
%    – Ihr technisches Team-Fazit
%    – Persönliches Fazit (je min. 1 Abschnitt pro Team-Mitglied)

%%%%%%%%%%%% TITLE PAGE END
    \newpage

    \pagenumbering{arabic}


    \section{Kapitel 1}
    \lipsum[10]


    \section{Kapitel 2}


\end{document}

