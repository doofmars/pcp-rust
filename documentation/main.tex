\documentclass[letterpaper,12pt]{article}
\usepackage{tabularx} % extra features for tabular environment
\usepackage{amsmath}  % improve math presentation
\usepackage{graphicx} % takes care of graphic including machinery
\usepackage[margin=1in,letterpaper]{geometry} % decreases margins
\usepackage{cite} % takes care of citations
\usepackage[final]{hyperref} % adds hyper links inside the generated pdf file
\hypersetup{
    colorlinks=true,       % false: boxed links; true: colored links
    linkcolor=blue,        % color of internal links
    citecolor=blue,        % color of links to bibliography
    filecolor=magenta,     % color of file links
    urlcolor=blue
}
%++++++++++++++++++++++++++++++++++++++++

% Do not indent second paragraph
\setlength{\parindent}{0pt}

\usepackage[ngerman]{babel}

\begin{document}

    \title{Programming Concepts \& Paradigms\\Rust}
    \author{Roman Schilter \& Jan-Henrik Preuss\\[0.4cm]{\small Betreuer: Marcel Baumann \& Ruedi Arnold}}
    \date{May 31, 2024}
    \maketitle

%Wenige Seiten (ca. 2-4) reichen durchaus, max. 5 (bei mehr
%gibt's tendenziell Abzug), Inhaltsverzeichnis nicht nötig

    \begin{abstract}
        % Beschreibe das dokument knapp
        Rust ist eine moderne Programmiersprache, die Sicherheit und Geschwindigkeit vereint.
        Sie wurde mit unterstützung von Mozilla 2010 entwickelt und ist Open Source.
        Rust hat den Anspruch, die Lücke zwischen Low-Level-Programmiersprachen wie C und C++ und High-Level-Programmiersprachen wie Java und Python zu schliessen.
        Dieses Dokument fasst die wichtigsten Konzepte und Paradigmen von Rust im Vergleich zu Java zusammen.
    \end{abstract}

%    Fokus auf wichtige/ interessante/ spezielle Sprach-
%Eigenschaften! (Was anders als bei Java?!...)
%§ Ergänzend zu den Folien
%– Falls interessant/relevant kurze Infos zu Vision,
%Geschichte & Verbreitung
%– Hauptteil: Die Sprache vorstellen (Ihre 3 bis 7
%Fokuspunkte, inkl. Verweise auf Ihren Demo-Code)


    \section{Sprachkonzepte}

    Hier werden die wichtigsten Sprachkonzepte von Rust vorgestellt.
    Um die Kürze des Dokuments zu gewährleisten, wurde auf eingebundenen Quellcode verzichtet.

    \subsection{Borrowing \& Move-Semantik}\label{subsec:borrowing-&-move-semantik}
    Die Borrowing \& Move-Semantik ist eines der wichtigsten Sprachfeatures von Rust.
    Mit ihr ist es möglich, die Speicherverwaltung auf elegante und abstrakte Weise auszulagern.
    Dadurch können typische Probleme der Speicherverwaltung umgangen werden\footnote
    {In unterschiedlichen projekten liegt der Anteil von Speicherfehlern bei \textgreater 60\% \url{https://alexgaynor.net/2020/may/27/science-on-memory-unsafety-and-security/}}.
    Dies führt zu sicherem Code der auch im Bare-Metal-Bereich eingesetzt werden kann.
    Variablen, die an andere Methoden übergeben werden, können als Reverenz ausgeliehen werden.
    Im Gegensatz zu C mit Pointern wird beim Kompilieren geprüft, ob die Verwendung der Referenz gültig ist.
    Die Crate `rust/example-borrow` zeigt funktionierende Beispiele, wie Variablen übergeben werden können.

    \subsection{Traits: bounds \& associated types}\label{subsec:traits:-bounds-&-associated-types}
    Rust kennt keine Vererbung.
    Structs, das Äquivalent einer Klasse in Rust, können nicht mit einer anderen Struct erweitert werden.
    Es gibt jedoch sogenannte Traits in Rust, die den Interfaces in Java sehr ähnlich sind.
    Traits können Methoden und optional auch eine Default-Implementierung für diese Methoden definieren.

    Der grosse Unterschied zwischen Traits und Interfaces ist jedoch, wie die Traits auf einer Struct implementiert werden.
    Anstatt bei der Definition des Structs angegeben zu werden, wird das Trait in einem separaten Codeblock implementiert.
    Das heisst, es ist völlig unabhängig von der Definition des Struct oder des Traits.
    Ein Beispiel dafür ist in der Crate `rust/example-traits`.

    Leider macht es das für den Entwickler nicht einfach, da nicht immer klar ist, welche Traits auf einem Struct implementiert sind.
    Vor allem nicht mit der folgenden Eigenart.

    \subsubsection{Trait Bounds}\label{subsubsec:trait-bounds}
    Anstatt ein Trait fest auf einem Struct zu implementieren, kann dies auch auf Basis der bereits implementierten Traits erfolgen.
    Beispielsweise kann TraitX auf allen Structs implementiert werden, die auch TraitY und TraitZ implementieren.
    Auf diese Weise können bestehende Structs oder Traits einfach um neue Funktionen erweitert werden.
    Es ist sogar möglich, ein Trait auf allem zu implementieren.
    So kann z.B.\ jedem String eine Funktion hinzugefügt werden.
    Ein Beispiel ist in der Crate `rust/example-trait-bounds` zu finden.

    \subsubsection{Traits mit associated types}\label{subsubsec:traits-associated-types}
    Traits unterstützen auch generische Typen.
    Damit kann ein Trait z.B.\ auf Struct1 mit einem Integer und auf Struct2 mit einem String implementiert werden.
    In Traits können auch assoziierte Typen definiert werden.
    Diese Typen werden bei der Implementierung gesetzt und erlauben es, wie Generics, die Trait-Typen unabhängig zu machen.
    Der Unterschied ist jedoch, dass der Type nur einmal gesetzt werden kann.
    Das heisst, wenn der Trait auf verschiedenen Strukturen mit String und mit Integer implementiert wird, kompiliert das Programm nicht.
    Dies ist nützlich, wenn man dies einschränken möchte.
    Zum Beispiel kann eine Bibliothek Traits mit assoziierten Typen ausliefern und muss sich nicht darum kümmern,
        dass die Typen zwischen den Implementierungen unterschiedlich sein können.
    Ein Beispiel für Traits mit variablen Typen findet sich in der Crate `rust/example-traits-associated-types`.

    \subsection{Typestate Programming}\label{subsec:typestate-programming)}
    Typestate Programming ist ein Pattern, mit dem der aktuelle State über den Typ abgebildet wird.
    In Rust bedeutet dies, dass jeder State ein Struct ist.
    Jedes Struct hat Methoden, die eine Instanz eines anderen Structs zurückgeben.
    Dadurch ist es möglich, von einem State in einen anderen zu wechseln.
    Jeder State kann auch zusätzliche Methoden haben, um zusätzliche Informationen über den State zu erhalten.
    Wenn sich die Structs in einem eigenen Modul befinden, kann mit dem Zugriffsmodifikator eingeschränkt werden, wie mit den States interagiert wird.
    So kann z.B.\ eingeschränkt werden, was der initiale Zustand ist und welche Transitionen gemacht werden können.
    In der Crate `rust/example-typestate-programming` gibt es ein Beispiel dafür.

    Rust eignet sich sehr gut für dieses Pattern, da durch die fixe Typisierung die States nicht manipuliert werden können.
    Dadurch kann der Entwickler sicherstellen, dass niemals ein invalider State auftreten kann.

    \subsection{Concurrency}\label{subsec:concurrency}
    \subsubsection{Threads}\label{subsubsec:threads}
    Threads können in Rust mit der Funktion \texttt{threads::spawn} gestartet werden.
    Als Argument muss eine Funktion ohne Parameter übergeben werden, die dann im Thread ausgeführt wird.
    Diese Funktion kann einen Wert zurückgeben oder mit dem Makro \texttt{panic!} einen Fehler ausgeben.
        \texttt{threads::spawn} gibt eine Instanz eines Threads zurück.
    Darauf kann die blockierende Funktion \texttt{join} aufgerufen werden, die ein Ergebnis zurückgibt.
    Anhand dieses Ergebnisses kann überprüft werden, ob der Thread "gepanicked" hat oder der Rückgabewert ausgelesen werden.
    Ein Beispiel hierfür findet sich in der Crate `rust/example-threads`.

    \subsubsection{Channels}\label{subsubsec:channels}
    Channels ermöglichen die Kommunikation zwischen zwei Threads in Rust.
    Dabei kann eine Variable von einem Thread zum anderen gesendet werden.
    Die Funktion \texttt{channel()} gibt ein Tupel mit einem Receiver und einem Sender zurück.
    Diese können dann jeweils einem Thread übergeben werden.

    An dieser Stelle kommt das Ownership-Konzept von Rust ins Spiel.
    Wenn ein Thread eine Variable an einen anderen Thread sendet, wird dieser zum Eigentümer der Variable.
    Das bedeutet, dass es zu einem Kompilierungsfehler kommt, wenn der erste Thread die Variable nach der Übergabe verwendet.
    Ein Beispiel dafür findet sich im Paket `rust/example-channels`.

    \subsection{Cargo: Test \& Build}\label{subsec:cargo:-test-&-build}
    Cargo ist das Build-Werkzeug für Rust, mit dem Standard-Supportfunktionen für die Entwicklung komfortabel ausgeführt werden können.
    In der Entwicklung ist das kontinuierliche Ausführen von Tests und Builds essentiell, um ein stabiles Produkt zu gewährleisten.
    Mit Cargo können die Tests gebündelt ausgeführt werden.
    Im Gegensatz zu Java ist Cargo direkt Teil der Installation und somit direkt verfügbar.
    Ein einfacher Import der \textit{reqwest} Bibliothek ist in der Crate `rust/example-reqwest` zu finden.

%    Ihr technisches Team-Fazit
    \section{Team-Fazit}\label{sec:team-fazit}
    Memory-Safety ist das Verkaufsargument für Rust.
    Das ist etwas, was andere Sprachen auf diesem Level nicht haben.
    Dies macht Rust zu einer modernen Sprache, was für die neue Generation von Entwicklern sehr attraktiv ist.

    Durch die Unterstützung im Linux-Kernel und die Gründung der Rust Foundation konnte Rust den Grundstein legen.
    Nun ist Rust auf dem Vormarsch und wird immer weiter verbreitet.
    Im Jahr 2020 war Rust noch auf Platz 35 der meistgenutzten Sprachen auf GitHub.
    Seit Ende 2023 liegt Rust nun auf Platz 20. \footnote{https://innovationgraph.github.com/global-metrics/programming-languages}.
    Derzeit verwenden etwas mehr als 13\% der befragten Entwickler Rust. \footnote{https://survey.stackoverflow.co/2023/\#technology}

% Persönliches Fazit (je min. 1 Abschnitt pro Team-Mitglied)
    \section{Persönliches Fazit}\label{sec:personliches-fazit}

    \subsection{Roman Schilter}\label{subsec:roman}
    Ich habe wenig bis keine Erfahrung mit Low-Level-Sprachen, daher war das Ownership-Konzept gewöhnungsbedürftig.
    Der Garbage Collector hat das Thema für mich immer sauber abstrahiert und ich musste mir nie Gedanken über Memory Leaks oder ähnliches machen.
    Daher bin ich froh, dass ich mir darüber auch bei Rust keine Gedanken machen muss.

    Die Syntax von Rust ist klar und intuitiv.
    Im Gegensatz zum Beispiel zu Clojure, dessen Syntax völlig anders ist als das, was ich bisher kannte.
    Einzig die Pipes bei Lambda-Ausdrücken waren etwas gewöhnungsbedürftig.

    Sich mit Rust auseinander zusetzen hat Spass gemacht und ich kann Rust empfehlen.
    Sollte ich jemals in den Genuss kommen, auf Lowlevel zu programmieren und die Wahl zwischen C, C++ oder Rust haben,
        würde ich Rust wählen.

    \subsection{Jan-Henrik Preuss}\label{subsec:jan}
    Mit Rust habe ich mich dank der Hilfe des Build-Tools Cargo direkt zurechtgefunden,
    der sehr guten Dokumentation über das online frei verfügbare Rust-Buch~\cite{rust-book}, sowie der exzellenten Fehlermeldungen und IDE-Unterstützung.
    All das gibt es auch für andere Programmiersprachen oder Frameworks, aber oft nicht so direkt und vor allem nicht so einheitlich.
    Bei der Recherche für diese Arbeit bin ich in das eine oder andere Rabbit Hole gestolpert, was mir gezeigt hat, wie groß und lebendig die Community von Rust ist.
    Ich bin gespannt was die Zukunft für Rust bringt und mein nächstes Projekt mit Mikrocontrollern werde ich auf jeden Fall mit Rust lösen.

%++++++++++++++++++++++++++++++++++++++++
% References section will be created automatically
% with inclusion of "thebibliography" environment
% as it shown below. See text starting with line
% \begin{thebibliography}{99}
% Note: with this approach it is YOUR responsibility to put them in order
% of appearance.

    \begin{thebibliography}{9}

        \bibitem{rust-book}
        \emph{The Rust Programming Language, 2nd Edition}.
        \newblock Steve Klabnik and Carol Nichols.
        \newblock No Starch Press, 2022.
        \newblock \url{https://doc.rust-lang.org}
    \end{thebibliography}


\end{document}
