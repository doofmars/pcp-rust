% Store the current frame an use this later as the final page number.
\newcounter{finalframe}
\setcounter{finalframe}{\value{framenumber}}

\begin{frame}
    \begin{center}
        \Huge Bonus slides
    \end{center}
\end{frame}

\begin{frame}[t]{Cargo create a new project}
    \codeoutput{code/08-cargo-new.txt}
    \only<2>{
        Wir erhalten die folgende Verzeichniss struktur:
        \dirtree{%
            .1 hello-world.
            .2 src.
            .3 main.rs.
            .2 Cargo.toml.
        }
    }
    \only<3>{
        Inhalt der main.rs:
        \lstinputlisting[language=Rust,
            label={lst:implementation_listing_hello-world-rs}]
        {../rust/hello-world/src/main.rs}
    }
    \only<4>{
        Inhalt der Cargo.toml:
        \lstinputlisting[language=Rust,
            label={lst:implementation_listing_hello-world-toml}]
        {../rust/hello-world/Cargo.toml}
    }
\end{frame}

\begin{frame}[t]{Cargo build and run}
    \codeoutput{code/08-cargo-build.txt}
    \only<1>{
        Starten der Anwendung durch ausführen der Binärdatei
        \codeoutput{code/08-cargo-run1.txt}
    }
    \only<2>{
        Starten mit cargo run
        \codeoutput{code/08-cargo-run2.txt}
    }
\end{frame}

\begin{frame}[t]{Cargo CI/CD}
    \framesubtitle{Automatisierung mit GitHub Actions}
    Wie auch in anderen programmiersprachen kann das Testen und Bauen mittels cargo in einer CI/CD Pipeline erfolgen.
    \mylistingHiglight{6}{16}{toml}{cargo-in-github}{\btLstHL<1>{}\btLstHL<2>{11,16}}{../.github/actions/build-rust/action.yml}
\end{frame}

%\begin{frame}[fragile]{Example slide with animations}
%    \framesubtitle{How to work with animations in latex?}
%    \begin{lstlisting}[
%        linebackgroundcolor={%
%        \btLstHL<1>{1-3}%
%        \btLstHL<2>{6,9}%
%        \btLstHL<3>{7}%
%        \btLstHL<4>{8}%
%        },
%        gobble=4,
%        label={lst:test}]
%      /**
%      * Prints Hello World.
%      **/
%      #include <stdio.h>
%
%      int main(void) {
%         printf("Hello World!");
%         return 0;
%      }
%    \end{lstlisting}
%
%    \only<1>{Shows on first page}
%    \only<1-2>{first and second page}
%    \only<3->{from-3}
%    \only<2>{second}
%    \onslide<3>{third}
%    \onslide<4>{fourth}
%    \pause
%    Then this comes on the last slide
%\end{frame}

% fool beamer on the total number of slides
\setcounter{framenumber}{\value{finalframe}}
