%% Listings Paket ------------------------------------------------------
%%% Doc: ftp://tug.ctan.org/pub/tex-archive/macros/latex/contrib/listings/listings-1.3.pdf
\usepackage{listings, ../latex-lib/listings-rust/listings-rust}
\usepackage{../latex-lib/lstlinebgrd/lstlinebgrd}

\definecolor{codegreen}{rgb}{0,0.6,0}
\definecolor{codegray}{rgb}{0.5,0.5,0.5}
\definecolor{codepurple}{rgb}{0.5,0,0.33}
\definecolor{codepurblue}{rgb}{0.16,0.0,1.0}
\definecolor{backcolour}{rgb}{1,0.94,0.70}

% I � umlauts
\lstset{literate=%
    {Ö}{{\"O}}1
    {Ä}{{\"A}}1
    {Ü}{{\"U}}1
    {ü}{{\"u}}1
    {ä}{{\"a}}1
    {ö}{{\"o}}1
}

\lstdefinelanguage{toml}{}

\lstset{
    basicstyle =\ttfamily\color{black}\small, % Standardschrift
    commentstyle=\color{codegreen},
    keywordstyle=\bfseries\color{codepurple},
    numberstyle=\tiny\color{codegray},
    stringstyle=\color{codepurblue},
    numbers = left,              % Ort der Zeilennummern
    tabsize=2,              % Groesse von Tabs
    breakatwhitespace=false,              % An Leerzeichen umbrechen
%showspaces=true,			  % Leerzeichen anzeigen
    backgroundcolor=\color{backcolour},      % % Hintergrundfarbe der Listings
    breaklines=true,
    captionpos=b,
    keepspaces=true,
    numbersep=5pt,
    showspaces=false,
    showstringspaces=false,
    showtabs=false,
}

% Code auschnitt importieren aus datei
% example:
%\mylisting{from}{to}{language}{label}{path}
%\mylisting{2}{2}{Rust}{hello-world}{../rust/src/main.rs}
\newcommand{\mylisting}[5]{
    \lstinputlisting[language=#3,
        firstnumber=#1,
        firstline=#1,
        lastline=#2,
        label={lst:implementation_listing_#4}]
    {#5}
}

% Code auschnitt importieren aus datei mit higlighting
% Requires btLstHL and lstlinebgrd with some fixes

% example:
%\mylistingHiglight{from}{to}{language}{label}{highlight-config}{path}
% % First frame no highlight and higlight lines 27-29 on second frame
%\mylistingHiglight{22}{31}{Rust}{fibonacci_recursive}{\btLstHL<1>{}\btLstHL<2>{27-29}}{../rust/exercise-prolog-w3-1/src/main.rs}
\newcommand{\mylistingHiglight}[6]{
    \lstinputlisting[language=#3,
        firstnumber=#1,
        firstline=#1,
        lastline=#2,
        linebackgroundcolor={#5},
        label={lst:implementation_listing_#4}]
    {#6}
}

%% End Listings Paket ------------------------------------------------------

%% Output console ----------------------------------------------------
\definecolor{outputbackground}{rgb}{0.9, 0.9, 0.9}

% Examples:
% % Good file
%\codeoutput{code/02-borrowing1.txt}
% % Error output file
%\codeoutput[red]{code/02-borrowing2.txt}

\newcommand{\codeoutput}[2][black]{%
    \lstinputlisting[
        basicstyle =\ttfamily\color{#1}\small, % Standardschrift
        backgroundcolor=\color{outputbackground},
        numbers = none,
        label={lst:code_#2}]
    {#2}
}

%% End Output console ----------------------------------------------------

%% Begin Listing higlight --------------------------------------------------
% Source https://tex.stackexchange.com/questions/8851/how-can-i-highlight-some-lines-from-source-code

%%%%%%%%%%%%%%%%%%%%%%%%%%%%%%%%%%%%%%%%%%%%%%%%%%%%%%%%%%%%%%%%%%%%%%%%%%%%%%
%
% \btIfInRange{number}{range list}{TRUE}{FALSE}
%
% Test in int number <number> is element of a (comma separated) list of ranges
% (such as: {1,3-5,7,10-12,14}) and processes <TRUE> or <FALSE> respectively
\usepackage{pgf, pgffor}

\makeatletter
\newcount\bt@rangea
\newcount\bt@rangeb

\newcommand\btIfInRange[2]{%
    \global\let\bt@inrange\@secondoftwo%
    \edef\bt@rangelist{#2}%
    \foreach \range in \bt@rangelist {%
        \afterassignment\bt@getrangeb%
        \bt@rangea=0\range\relax%
        \pgfmathtruncatemacro\result{ ( #1 >= \bt@rangea) && (#1 <= \bt@rangeb) }%
        \ifnum\result=1\relax%
        \breakforeach%
        \global\let\bt@inrange\@firstoftwo%
        \fi%
    }%
    \bt@inrange%
}
\newcommand\bt@getrangeb{%
    \@ifnextchar\relax%
        {\bt@rangeb=\bt@rangea}%
        {\@getrangeb}%
}
\def\@getrangeb-#1\relax{%
    \ifx\relax#1\relax%
    \bt@rangeb=100000%   \maxdimen is too large for pgfmath
    \else%
    \bt@rangeb=#1\relax%
    \fi%
}

%%%%%%%%%%%%%%%%%%%%%%%%%%%%%%%%%%%%%%%%%%%%%%%%%%%%%%%%%%%%%%%%%%%%%%%%%%%%%%
%
% \btLstHL<overlay spec>{range list}
% Usage in lstlisting:
% linebackgroundcolor={%
%        \btLstHL<1>{}% No highliting here
%        \btLstHL<2>{3}%
%        }
%
% TODO BUG: \btLstHL commands can not yet be accumulated if more than one overlay spec match.
%
\newcommand<>{\btLstHL}[1]{%
    \only#2{\btIfInRange{\value{lstnumber}}{#1}{\color{red!30}\def\lst@linebgrdcmd{\color@block}}{\def\lst@linebgrdcmd####1####2####3{}}}%
}%
\makeatother
%% End Listing higlight ----------------------------------------------------
