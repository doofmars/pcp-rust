% Final slides here
\begin{frame}[t]{Kontrollfragen}
    \begin{tcolorbox}[purplebox]
        \begin{enumerate}
            \item Werdet ihr in Zukunft die Programmiersprache Rust verwenden?
            \item Ist Rusts Borrow Checker im Compiler der überfürsorgliche Elternteil meines Codes?
            \item Wird Rust irgendwann `oxidieren' und eine neue Programmiersprache werden, oder bleibt es beständig?
            \item Kompiliert dieser Code? Wenn ja, was gibt er aus?\\
            \lstinputlisting[backgroundcolor=\color{myPurpleBackground},frame=none,numbers=none]{code/final.rs}
        \end{enumerate}
    \end{tcolorbox}
\end{frame}

%% REFERENCES
% allowframebreaks: creates multiple slides if it is to long for one
\begin{frame}[allowframebreaks]{References}
    \begin{thebibliography}{}
        \setbeamertemplate{bibliography item}[book]
        \bibitem{rust-book}
        \emph{The Rust Programming Language, 2nd Edition}.
        \newblock Steve Klabnik and Carol Nichols.
        \newblock No Starch Press, 2022.
        \newblock \url{https://doc.rust-lang.org}

%        \setbeamertemplate{bibliography item}[article]
%        \bibitem{FaceNet}
%        Schroff, Florian, Dmitry Kalenichenko, and James Philbin.
%        \newblock \emph{FaceNet: A Unified Embedding for Face Recognition and Clustering}, 2015.
%        \newblock \url{https://arxiv.org/abs/1503.03832}

%        \setbeamertemplate{bibliography item}[online]
%        \bibitem{DCASE}
%        Dekkers, Lauwereins, Thoen et al.
%        \newblock \emph{DCASE Challenge 2018 Task 5}, 2018.
%        \newblock \url{http://dcase.community/challenge2018/task-monitoring-domestic-activities}
    \end{thebibliography}
\end{frame}
