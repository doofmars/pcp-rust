% Concurrency: Threads, Spawn, Results, Channel, panic!

\begin{frame}[fragile,t]{Threads}
    Neuer Thread:
    \begin{lstlisting}[language=Rust,escapechar=@,label={lst:threads1-1}]
let thread = thread::spawn(|| {
    // do stuff
});\end{lstlisting}

    \note<1> {
        \begin{itemize}
            \item Rust kann auch nebenläufigkeit.
            \item Mit thread::spawn kann ein neuer Thread gestartet werden.
            \item Diesem gibt man ein Lambda mit, das im Thread ausgeführt wird.
            \item Der Thread wird direkt beim spawn gestartet.
            \item Es gibt keine Möglichkeit einen Thread zu erzeugen und später zu starten.
            \item Braucht es auch nicht, da das Lambda, einfach in eine Variable geschrieben werden kann.
        \end{itemize}
    }

    \pause
    Thread join:
    \begin{lstlisting}[language=Rust,escapechar=@,label={lst:threads1-2}]
thread.join();
\end{lstlisting}

    \note<2>{
        \begin{itemize}
            \item Mit join auf den Abschluss des Threads gewartet werden
            \item Dabei erhält man ein Resultat aus dem der Rückgabewert oder ein Fehler aus dem Lambda entgegen genommen werden kann
        \end{itemize}
    }

\end{frame}

\begin{frame}[fragile,t]{Thread Channels I}
    Channel erstellen:
    \begin{lstlisting}[language=Rust,escapechar=@,label={lst:thread-channels-1-1}]
let (sender, receiver) = channel();
\end{lstlisting}

    \note<1>{
        \begin{itemize}
            \item Threads können auch zwischeneinander Kommunizieren während beide noch laufen.
            \item Dazu können channels verwendet werden.
            \item So erstellt man einen Channel: Dabei erhält man jeweils einen sender und einen receiver für den channel.
        \end{itemize}
    }

    \pause
    Daten senden:
    \begin{lstlisting}[language=Rust,escapechar=@,label={lst:thread-channels-1-2}]
let sending_thread = thread::spawn(move || {
    sender.send("important data");
});\end{lstlisting}

    \note<2>{
        So kann mit dem sender Daten an den receiver geschickt werden
    }

    \pause
    Daten empfangen:
    \begin{lstlisting}[language=Rust,escapechar=@,label={lst:thread-channels-1-3}]
let receiving_thread = thread::spawn(move || {
     let data = receiver.recv();
});\end{lstlisting}

    \note<3>{
        \begin{itemize}
            \item So können die Daten mit dem receiver empfangen werden
            \item Das move-keyword vor dem lambda erlaubt es innerhalb des Threads den sender/receiver zu verwenden
            \item Diese variablen werden in den Thread gemoved
        \end{itemize}

        \begin{itemize}
            \item Es können mehr als nur Strings über channels gesandt werden, structs gehen zum Beispiel auch.
            \item Dabei sollten jetzt die Alarmglocken klingeln => was passiert wenn ich den gesendeten Wert verändere.
        \end{itemize}
    }
\end{frame}

\begin{frame}[fragile,t]{Thread Channels II}
    \begin{lstlisting}[language=Rust,escapechar=@,label={lst:thread-channels-2-1}]
let sending_thread = thread::spawn(move || {
    let data = Data {
        ...
    };

    sender.send(data);

    println!("Data: {}", data);
});\end{lstlisting}

    \note{
        \begin{itemize}
            \item Hier kann man jetzt sehen warum das Ownership Konzept von Rust für threads ein Vorteil ist
            \item Die Variable wird beim Send and den anderen Thread geschickt, nicht nur der Wert
            \item Lass uns das mal im Code aunschauen => DEMO
        \end{itemize}
    }

    \pause
    $\longrightarrow$ DEMO
\end{frame}
